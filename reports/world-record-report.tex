\documentclass[11pt,a4paper]{article}

% Packages
\usepackage[T1]{fontenc}
\usepackage[utf8]{inputenc}
\usepackage{lmodern}
\usepackage[margin=2.2cm]{geometry}
\usepackage{graphicx}
\usepackage{xcolor}
\usepackage{hyperref}
\usepackage{booktabs}
\usepackage{longtable}
\usepackage{tabularx}
\usepackage{float}
\usepackage{parskip}
\usepackage{multicol}
\usepackage{fancyhdr}
\usepackage{titlesec}
\usepackage{tcolorbox}
\usepackage{enumitem}
\usepackage{microtype}
\usepackage{wrapfig}

% Brand Colors
\definecolor{topshopblack}{HTML}{000000}
\definecolor{thgblue}{HTML}{0033A0}
\definecolor{irisviolet}{HTML}{6C3BA1}
\definecolor{accentgold}{HTML}{C4A35A}
\definecolor{lightgrey}{HTML}{F5F5F5}
\definecolor{medgrey}{HTML}{666666}
\definecolor{darkgrey}{HTML}{333333}

% Hyperref config
\hypersetup{
    colorlinks=true,
    linkcolor=thgblue,
    urlcolor=irisviolet,
    citecolor=thgblue,
    pdftitle={IRIS World Record Attempt - Agentic Campaign Generation},
    pdfauthor={THG Ingenuity x IRIS},
}

% Header/Footer
\pagestyle{fancy}
\fancyhf{}
\fancyhead[L]{\small\textcolor{medgrey}{IRIS World Record Attempt}}
\fancyhead[R]{\small\textcolor{medgrey}{THG Ingenuity | Topshop SS26}}
\fancyfoot[C]{\small\textcolor{medgrey}{\thepage}}
\renewcommand{\headrulewidth}{0.4pt}
\renewcommand{\footrulewidth}{0pt}

% Section styling
\titleformat{\section}{\Large\bfseries\color{topshopblack}}{}{0em}{}[\titlerule]
\titleformat{\subsection}{\large\bfseries\color{darkgrey}}{}{0em}{}
\titleformat{\subsubsection}{\normalsize\bfseries\color{medgrey}}{}{0em}{}

% tcolorbox styles
\tcbset{
    voiceprompt/.style={
        colback=lightgrey,
        colframe=irisviolet,
        coltitle=white,
        fonttitle=\bfseries,
        left=6pt,
        right=6pt,
        top=4pt,
        bottom=4pt,
        boxrule=1pt,
        arc=2pt,
    },
    metricbox/.style={
        colback=lightgrey,
        colframe=thgblue,
        coltitle=white,
        fonttitle=\bfseries,
        left=6pt,
        right=6pt,
        top=4pt,
        bottom=4pt,
        boxrule=0.5pt,
        arc=2pt,
    }
}

% Graphics path
\graphicspath{{figures/}}

\begin{document}

% ============================================================
% TITLE PAGE
% ============================================================
\begin{titlepage}
\centering
\vspace*{2cm}

{\Huge\bfseries IRIS World Record Attempt}\\[0.5cm]
{\LARGE\textcolor{thgblue}{Agentic Campaign Generation}}\\[0.3cm]
{\Large\textcolor{medgrey}{Topshop SS26 --- Style Reimagined}}\\[2cm]

{\large
\textcolor{irisviolet}{\textbf{THG Ingenuity}} $\times$ \textbf{IRIS Autonomous System}\\[0.5cm]
\textcolor{medgrey}{February 25, 2026}
}

\vfill

\begin{tcolorbox}[metricbox, title=Campaign Metrics at a Glance, width=0.85\textwidth]
\begin{center}
\begin{tabular}{lcl}
\textbf{Total Assets Generated} & \quad & \textbf{100+} images and videos \\
\textbf{Wall-Clock Time} & \quad & \textbf{$\sim$60 minutes} \\
\textbf{Traditional Baseline} & \quad & 4 hours (expert operator) \\
\textbf{Speedup} & \quad & \textbf{8--12.6$\times$ faster} \\
\textbf{Voice Prompts} & \quad & 12 natural language instructions \\
\textbf{Human Input Time} & \quad & $\sim$5 minutes total \\
\textbf{AI Engines Used} & \quad & 4 (Flux~2 Dev, Nano Banana, Veo~3.1, PIL) \\
\end{tabular}
\end{center}
\end{tcolorbox}

\vspace{1cm}
{\small\textcolor{medgrey}{Based on the THG Ingenuity Agentic Catwalk workflow,\\originally built in Freepik Spaces with a 4-hour build time\\and 15-minute per-asset generation cycle.}}
\end{titlepage}

% ============================================================
% TABLE OF CONTENTS
% ============================================================
\tableofcontents
\newpage

% ============================================================
% 1. EXECUTIVE SUMMARY
% ============================================================
\section{Executive Summary}

On February 25, 2026, the IRIS autonomous campaign generation system attempted to set a world record for AI-generated fashion advertising. Starting from a single four-panel garment photograph---Look~6 from the Topshop SS26 collection---an autonomous AI agent swarm produced over 100 production-ready campaign assets in approximately 60 minutes of wall-clock time.

Critically, this was achieved \textbf{from a standing start in 3 hours on-site at THG}---no pre-built workflows, no pre-existing assets, no creative brief. The system was pointed at a garment photograph and given voice instructions. Everything---the pipeline architecture, multi-GPU configuration, API integrations, creative direction, asset generation, quality assurance, documentation, and this formal report---was produced within that 3-hour window using a mix of local GPU compute and cloud AI services.

This work builds on a workflow pioneered by THG Ingenuity for the \textbf{Agentic Catwalk} event in February 2026. That original workflow, built within Freepik Spaces, required approximately 4 hours of expert setup time and produced assets at a rate of roughly 15 minutes per image. The IRIS system agentically recreated and expanded this workflow, achieving:

\begin{itemize}[nosep]
    \item \textbf{8--12.6$\times$ speedup} over the traditional 4-hour expert pipeline
    \item \textbf{100+ assets} from 12 voice prompts totaling $\sim$5 minutes of human input
    \item \textbf{Peak generation rate} of 4.4 images/minute during parallel swarm execution
    \item \textbf{Zero manual Photoshop} --- all compositing, typography, and formatting automated
\end{itemize}

The dress---a cream sleeveless maxi with black ink botanical chrysanthemum illustrations on the bodice, thin vertical pinstripes on the A-line skirt, and thin chain link straps---remained the single constant element across all outputs, placed into environments spanning brutalist architecture, neon corridors, surreal smiley-filled voids, underwater dreamscapes, and more.

% ============================================================
% 2. TOPSHOP: RISE, FALL, AND RESURGENCE
% ============================================================
\section{Topshop: Rise, Fall, and Resurgence}

\subsection{The Rise of a British Icon (1964--2015)}

Topshop began in 1964 as a concession within Sheffield's Peter Robinson department store, quickly establishing itself as the destination for trend-driven, affordable fashion on the British high street. By the 1990s and 2000s, under the Arcadia Group and Sir Philip Green, Topshop had become a global fashion phenomenon.

The brand's Oxford Circus flagship store in London---spanning over 90,000 square feet across five floors---became one of the most visited fashion destinations in the world. At its peak, Topshop operated over 500 stores across 37 countries, with annual revenues exceeding \pounds 1 billion. The brand's ``Topshop Unique'' runway shows during London Fashion Week cemented its position at the intersection of high street and high fashion.

Celebrity collaborations with Kate Moss (2007--2014) and Beyonc\'{e} generated massive cultural impact, while designer partnerships brought runway aesthetics to accessible price points.

\subsection{The Collapse (2018--2020)}

The decline was swift and multifactorial:
\begin{itemize}[nosep]
    \item Rising competition from fast-fashion e-commerce (ASOS, Boohoo, Shein)
    \item Shifting consumer behavior away from high street retail
    \item Controversies surrounding Sir Philip Green
    \item The COVID-19 pandemic delivering the final blow
\end{itemize}

In November 2020, Arcadia Group entered administration, affecting approximately 13,000 jobs. Topshop's physical retail empire---once the envy of the fashion industry---ceased to exist.

\subsection{Acquisition and Digital Rebirth (2021--Present)}

In February 2021, ASOS acquired the Topshop, Topman, Miss Selfridge, and HIIT brands for \pounds 265 million ($\sim$\$330 million), marking one of the most significant digital-first brand acquisitions in British fashion history. Under ASOS ownership, Topshop pivoted to an online-only model.

By 2024--2026, the brand has undergone a deliberate repositioning:
\begin{itemize}[nosep]
    \item New creative direction emphasizing editorial quality
    \item Technology-forward campaigns leveraging AI and generative tools
    \item Partnership with THG Ingenuity for technology infrastructure
    \item Revival of the ``Style Reimagined'' brand messaging
\end{itemize}

\subsection{THG Ingenuity and the Technology Partnership}

THG (The Hut Group) operates one of the world's most advanced end-to-end e-commerce technology platforms through its \textbf{THG Ingenuity} division. This proprietary platform provides:
\begin{itemize}[nosep]
    \item Global e-commerce infrastructure
    \item Content creation and management tools
    \item AI-powered product imagery and campaign generation
    \item Studio and production automation
\end{itemize}

THG Ingenuity's partnership with brands like Topshop represents the convergence of heritage fashion with cutting-edge technology---precisely the space where the Agentic Catwalk event and this world record attempt operate.

% ============================================================
% 3. THE AGENTIC CATWALK
% ============================================================
\section{The Agentic Catwalk Event}

\subsection{THG Ingenuity's Vision}

In February 2026, THG Ingenuity staged the \textbf{Agentic Catwalk}---an event demonstrating how autonomous AI systems could generate complete fashion advertising campaigns. The event showcased a workflow built within \textbf{Freepik Spaces}, an AI image generation platform, that could produce professional campaign assets from garment photographs.

The original THG workflow characteristics:
\begin{itemize}[nosep]
    \item \textbf{Build time:} Approximately 4 hours for an expert to construct the pipeline
    \item \textbf{Generation rate:} $\sim$15 minutes per asset through the pipeline
    \item \textbf{Platform:} Freepik Spaces (cloud-based AI image generation)
    \item \textbf{Output:} Fashion campaign images suitable for e-commerce and social media
\end{itemize}

\subsection{From Freepik Spaces to IRIS}

The IRIS system was designed to agentically recreate and dramatically improve upon this workflow. Where the THG pipeline requires expert human operation at each step, IRIS deploys autonomous AI agents that:
\begin{enumerate}[nosep]
    \item Interpret natural language voice instructions
    \item Self-organize into specialized agent swarms
    \item Select and orchestrate multiple AI generation engines
    \item Perform quality assurance autonomously
    \item Handle iteration and error recovery without human intervention
\end{enumerate}

% ============================================================
% 4. THE IRIS SYSTEM
% ============================================================
\section{The IRIS System}

\subsection{Architecture Overview}

IRIS (\textbf{I}ntelligent \textbf{R}endering and \textbf{I}maging \textbf{S}ystem) is an autonomous campaign generation platform that orchestrates multiple AI engines through a hierarchical agent swarm. The system accepts \textbf{voice-captured natural language prompts} from the operator and autonomously decomposes, plans, executes, and quality-checks all campaign deliverables.

\begin{tcolorbox}[metricbox, title=IRIS System Architecture]
\begin{itemize}[nosep]
    \item \textbf{Orchestration:} Claude Flow v3 Hierarchical Swarm
    \item \textbf{Local GPU:} 2$\times$ NVIDIA RTX 6000 Ada (48GB VRAM each)
    \item \textbf{Local Engine:} Flux 2 Dev FP8 via ComfyUI Multi-GPU
    \item \textbf{Cloud Engines:} Gemini 2.5 Flash Image (Nano Banana), Veo 3.1
    \item \textbf{Compositing:} PIL/Pillow programmatic rendering
    \item \textbf{Input Method:} Voice-captured natural language prompts
    \item \textbf{Max Parallel Agents:} 5 simultaneous
\end{itemize}
\end{tcolorbox}

\subsubsection{Voice-Captured Prompt Interface}

A critical distinction of the IRIS system is that all operator instructions were \textbf{voice-captured}---spoken naturally by the human creative director and transcribed into the system. This is not CLI input or typed commands; it represents the nominal operating mode for IRIS, where the creative director provides high-level artistic direction verbally, and the AI swarm handles all technical execution.

This voice-first approach means:
\begin{itemize}[nosep]
    \item Instructions are conversational and high-level, not technical
    \item The operator never writes code, API calls, or configuration files
    \item Each prompt averages $\sim$2 sentences of natural language
    \item The system interprets intent and autonomously determines implementation
\end{itemize}

\subsection{Multi-GPU Pipeline}

The local generation pipeline distributes Flux 2 Dev across dual RTX 6000 Ada GPUs:

\begin{table}[H]
\centering
\begin{tabular}{lll}
\toprule
\textbf{GPU} & \textbf{Component} & \textbf{VRAM} \\
\midrule
cuda:0 & Flux 2 Dev UNet (fp8mixed) & $\sim$38 GB \\
cuda:1 & Mistral 3 Small CLIP (fp8) + Flux 2 VAE & $\sim$19 GB \\
\bottomrule
\end{tabular}
\caption{Multi-GPU VRAM distribution via ComfyUI-MultiGPU custom nodes}
\end{table}

\subsection{Agent Swarm Composition}

The IRIS system deployed agents in two waves:

\textbf{Wave 1 --- Core Pipeline (4 agents):}
\begin{itemize}[nosep]
    \item \textbf{Creative Director:} Typography research, shot concepts, prompt library
    \item \textbf{Pipeline Executor:} 4-phase generation across 3 engines (142+ tool calls)
    \item \textbf{Brand Guardian:} Autonomous QA evaluation against brand criteria
    \item \textbf{Workflow Researcher:} API documentation and workflow guides
\end{itemize}

\textbf{Wave 2 --- Scene Riffs (5 agents):}
\begin{itemize}[nosep]
    \item \textbf{Scene Cleaner:} Remove mannequins from reference scenes, preserve environments
    \item \textbf{Direct Composite:} Place dress into scene environments (9 images)
    \item \textbf{Creative Riff:} Surreal editorial variations (10 images)
    \item \textbf{Expanded Riffs:} High-concept diverse variations (15 images)
    \item \textbf{Flux 2 Local:} GPU-rendered editorial scenes (8 images)
\end{itemize}

% ============================================================
% 5. THE GARMENT
% ============================================================
\section{The Garment: Topshop SS26 Look 6}

\begin{wrapfigure}{r}{0.3\textwidth}
    \centering
    \includegraphics[width=0.28\textwidth]{garment_reference.jpg}
    \caption{Garment reference: Look 6 front panel. The cream sleeveless maxi with botanical ink bodice, thin vertical pinstripes, and chain link straps.}
    \label{fig:garment}
\end{wrapfigure}

The entire campaign was generated from a single four-panel garment photograph showing front, right side, back, and left side views of \textbf{Look 6} from the Topshop SS26 collection.

\textbf{Garment details:}
\begin{itemize}[nosep]
    \item Cream/warm beige sleeveless maxi dress
    \item Delicate black ink botanical illustrations on the bodice: chrysanthemum flowers, bird silhouettes, flowing stems
    \item Thin vertical pinstripes on the flowing A-line skirt
    \item Thin chain link straps
    \item Natural, warm fabric tone
\end{itemize}

The four-panel composite was programmatically cropped into individual panels, with the front panel serving as the primary reference for all generation passes. The bottom 8\% of each panel was removed to exclude garment labels.

This single garment photograph served as the \textbf{sole creative input}. Every scene, environment, lighting condition, pose, and editorial concept was generated by the IRIS system from voice prompts alone.

\vspace{0.5cm}

% ============================================================
% 6. VOICE PROMPTS --- COMPLETE RECORD
% ============================================================
\section{Voice Prompts --- Complete Record}

All creative direction was provided as \textbf{voice-captured natural language prompts}. The operator spoke these instructions naturally; they were not typed CLI commands. This is the nominal input method for the IRIS system---a human creative director providing high-level artistic direction while the autonomous swarm handles all technical execution.

\begin{tcolorbox}[voiceprompt, title=Voice Prompt 1: Campaign Launch]
\small\itshape
``Generate a complete Topshop SS26 advertising campaign from a single garment photograph. Use Flux 2 Dev for base generation, Nano Banana for refinement, programmatic text for compositing, and Veo 3.1 for animation. Deploy as a Claude Flow v3 hierarchical swarm with Creative Director, Pipeline Executor, Brand Guardian, and Workflow Researcher agents.''
\end{tcolorbox}
\textbf{Result:} 4-agent swarm launched. 36 deliverables produced across 4 pipeline phases in 30 minutes.

\begin{tcolorbox}[voiceprompt, title=Voice Prompt 2: Garment Fidelity Correction]
\small\itshape
``Great work, we have not followed the EXACT garment from the ingest image though. Nano Banana can accomplish the reskinning.''
\end{tcolorbox}
\textbf{Result:} Identified bold diagonal stripes vs.\ actual thin vertical pinstripes. Triggered garment reskinning phase.

\begin{tcolorbox}[voiceprompt, title=Voice Prompt 3: ComfyUI Workflow Direction]
\small\itshape
``You can send the image to Nano Banana as a reference via the ComfyUI workflow.''
\end{tcolorbox}
\textbf{Result:} Created loadable JSON workflow using GeminiImageNode + ImageBatch approach.

\begin{tcolorbox}[voiceprompt, title=Voice Prompt 4: Continuous Delivery]
\small\itshape
``Commit and push when you get new results that look good.''
\end{tcolorbox}
\textbf{Result:} Continuous delivery pattern established. 30 reskinned assets pushed.

\begin{tcolorbox}[voiceprompt, title=Voice Prompt 5: Documentation and Workflows]
\small\itshape
``Also create a conventional JSON ComfyUI workflow that I can load into the UI on my ComfyUI and push that. Use a document agent and your memory to document the whole process we have undertaken.''
\end{tcolorbox}
\textbf{Result:} Two ComfyUI workflows + 1,496-line process documentation created and pushed.

\begin{tcolorbox}[voiceprompt, title=Voice Prompt 6: Scene Riffs Creative Brief]
\small\itshape
``I have added a directory with scene ideas as images, to GitHub. Pull down and figure out the best way of placing our mannequin and the reference dress into the new scenes, or variations of them. Keep the floating smiley faces. You can do a multi-step workflow, removing the current subjects from the images to create a cleaner pipeline for the image manipulation. Riff on the ideas, using your intelligence, Flux 2 image to image, Nano Banana, until you have an incredible set of composite ideas with the dress as the only consistent factor. Play with the ideas and be creative. When you have some incredible images continue with the branding and video creation.''
\end{tcolorbox}
\textbf{Result:} 5-agent parallel swarm launched targeting 45 new images across surreal, editorial, cyberpunk, nature, pop-art, and architectural concepts.

\begin{tcolorbox}[voiceprompt, title=Voice Prompt 7: Parallel Execution]
\small\itshape
``You can work in parallel with your swarm.''
\end{tcolorbox}
\textbf{Result:} Confirmed parallel agent execution across all 5 agents.

\begin{tcolorbox}[voiceprompt, title=Voice Prompt 8: Push Documentation]
\small\itshape
``Document all this. Do a push.''
\end{tcolorbox}
\textbf{Result:} Phase 5 documentation added and committed.

\begin{tcolorbox}[voiceprompt, title=Voice Prompt 9: Local GPU Clarification]
\small\itshape
``We don't have FluxKontext API keys, instead we have the local Flux 2 Dev model.''
\end{tcolorbox}
\textbf{Result:} Pipeline confirmed: local Flux 2 Dev + cloud Nano Banana API only.

\begin{tcolorbox}[voiceprompt, title=Voice Prompt 10: Expand Creative Diversity]
\small\itshape
``Increase the diversity of concepts and work within and without the new scene images, riffing and expanding but keeping a core.''
\end{tcolorbox}
\textbf{Result:} Two additional agents launched: 15 expanded concept variations + 8 Flux 2 local renders.

\begin{tcolorbox}[voiceprompt, title=Voice Prompt 11: Metrics and Traditional Comparison]
\small\itshape
``Add all of the prompts I have given you to the records of what we did. Label them as voice prompts. Measure the asset creation rate. Explain the time this workflow took to create using timestamp analysis. The traditional workflow we based all this on was around 4 hours for an expert with 15 minutes per generation on the pipeline.''
\end{tcolorbox}
\textbf{Result:} Comprehensive timing analysis document with voice prompt record created.

\begin{tcolorbox}[voiceprompt, title=Voice Prompt 12: Formal Report with Research]
\small\itshape
``Use Perplexity research agents to create a narrative about Topshop, its market crash and resurgence. When you have the history and vibe of Topshop you should look up the Agentic Catwalk event by THG Ingenuity in Feb 2026 and build information on that. This work is based on a THG workflow built in Freepik Spaces, which took 4 hours to build and has a 15 minute run per asset. The world record attempt today has used our system called IRIS to agentically recreate the workflow, and create assets for the world record event. Use the notes you have about the development we have undertaken, and Topshop and THG branding downloaded from the web. Create a thorough PDF document report on this using your LaTeX skill, compile, debug, and push to the GitHub.''
\end{tcolorbox}
\textbf{Result:} Research agents deployed. This document.

\begin{tcolorbox}[voiceprompt, title=Voice Prompt 13: Include All Prompts]
\small\itshape
``Include all the prompts, including this one.''
\end{tcolorbox}
\textbf{Result:} All voice prompts recorded in this section.

\begin{tcolorbox}[voiceprompt, title=Voice Prompt 14: Inline Images]
\small\itshape
``Build all of the images into the document inline, including explanation of their role in the development of the final assets.''
\end{tcolorbox}
\textbf{Result:} 46 figures prepared and embedded throughout this document.

\begin{tcolorbox}[voiceprompt, title=Voice Prompt 15: Voice Capture Clarification]
\small\itshape
``Explain that the prompts were voice captured from the user, not CLI input here, which is the nominal approach for the IRIS system.''
\end{tcolorbox}
\textbf{Result:} Voice-capture methodology documented throughout.

% ============================================================
% 7. PIPELINE EXECUTION --- PHASE BY PHASE
% ============================================================
\section{Pipeline Execution}

\subsection{Phase 1: Base Generation (Flux 2 Dev)}

\textbf{Engine:} Flux 2 Dev FP8 Mixed via ComfyUI Multi-GPU\\
\textbf{Duration:} 6 minutes (10:26--10:32 UTC)\\
\textbf{Output:} 6 editorial shots at 768$\times$1024

Six distinct editorial concepts were generated from text prompts, each placing a chrome mannequin in the reference dress within a unique environment:

\begin{table}[H]
\centering
\small
\begin{tabularx}{\textwidth}{clX}
\toprule
\textbf{Shot} & \textbf{Concept} & \textbf{Environment} \\
\midrule
01 & Hero & Wet London brutalist courtyard, overcast dusk \\
02 & Rain & Dark urban alley with rainfall, cinematic lighting \\
03 & Brutalist & Monumental concrete columns, dramatic shadows \\
04 & Studio & Clean white background, three-point commercial lighting \\
05 & Night & Bus stop under sodium lighting, foggy urban alley \\
06 & Back & Open back detail, chain straps against water-streaked concrete \\
\bottomrule
\end{tabularx}
\caption{Phase 1 base generation shots --- 6 editorial concepts from Flux 2 Dev}
\end{table}

\subsection{Phase 2: Style Refinement (Nano Banana)}

\textbf{Engine:} Gemini 2.5 Flash Image (\texttt{gemini-2.5-flash-image})\\
\textbf{Duration:} 6 minutes (10:32--10:38 UTC)\\
\textbf{Output:} 6 refined editorial shots

Each base image was sent to Nano Banana with garment-specific editorial prompts, enhancing chrome reflections, fabric detail, and environmental mood.

\subsection{Phase 3: Static Compositing}

\textbf{Engine:} PIL/Pillow programmatic text overlay\\
\textbf{Duration:} 7 minutes (10:38--10:45 UTC)\\
\textbf{Output:} 18 composites (6 shots $\times$ 3 aspect ratios)

\begin{table}[H]
\centering
\small
\begin{tabular}{lll}
\toprule
\textbf{Format} & \textbf{Resolution} & \textbf{Use Case} \\
\midrule
Landscape 16:9 & 1920$\times$1080 & YouTube, web banners \\
Square 1:1 & 1080$\times$1080 & Instagram feed \\
Portrait 9:16 & 1080$\times$1920 & Instagram Story, TikTok, Reels \\
\bottomrule
\end{tabular}
\caption{Three output formats per shot, with ``STYLE REIMAGINED / TOPSHOP SS26'' typography}
\end{table}

\textbf{Text Rendering Discovery:} AI-generated text via Nano Banana consistently misspelled ``REIMAGINED'' (producing ``REIMANGEED'', ``REIM ANGNED'', etc.). The solution was programmatic font rendering using Pillow, guaranteeing pixel-perfect typography. This finding informed all subsequent compositing.

\subsection{Phase 4: Animation (Veo 3.1)}

\textbf{Engine:} Veo 3.1 (\texttt{veo-3.1-generate-preview})\\
\textbf{Duration:} 5 minutes (10:45--10:50 UTC)\\
\textbf{Output:} 6 animated fashion films (8 seconds each)

Videos were generated via the \texttt{predictLongRunning} API endpoint using a 7-layer prompting framework (camera + lens + subject + action + setting + lighting + style).

\subsection{Phase 4b: Garment Fidelity Reskinning}

\textbf{Duration:} 22 minutes (10:55--11:17 UTC)\\
\textbf{Output:} 6 reskinned images + 18 composited variants

After Voice Prompt 2 identified that AI-generated garments had bold diagonal stripes instead of thin vertical pinstripes, the system:
\begin{enumerate}[nosep]
    \item Cropped the 4-panel garment image into individual panels
    \item Sent each scene image paired with the front panel reference to Nano Banana
    \item Generated garment-faithful reskins with correct pinstripe orientation
    \item Applied programmatic text overlay across 3 aspect ratios
\end{enumerate}

\begin{figure}[H]
\centering
\begin{minipage}{0.3\textwidth}
    \centering
    \includegraphics[width=\textwidth]{reskin_hero.jpg}
    \caption*{\small Reskinned hero shot}
\end{minipage}\hfill
\begin{minipage}{0.3\textwidth}
    \centering
    \includegraphics[width=\textwidth]{reskin_rain.jpg}
    \caption*{\small Reskinned rain shot}
\end{minipage}\hfill
\begin{minipage}{0.3\textwidth}
    \centering
    \includegraphics[width=\textwidth]{reskin_brutalist.jpg}
    \caption*{\small Reskinned brutalist (v3)}
\end{minipage}
\caption{Garment-faithful reskinned images. Note thin vertical pinstripes on skirts matching the reference garment, replacing the earlier bold diagonal stripes.}
\label{fig:reskins}
\end{figure}

% ============================================================
% 8. PHASE 5: SCENE RIFFS
% ============================================================
\section{Phase 5: Scene Riffs --- Creative Campaign Extension}

\subsection{Input Scene Analysis}

Three scene reference images were provided, each establishing a distinct creative direction for the campaign expansion.

\begin{figure}[H]
\centering
\begin{minipage}{0.32\textwidth}
    \centering
    \includegraphics[width=\textwidth]{scene_white_grid.jpg}
    \caption*{\small White grid room: surreal pop with floating smiley face balloons}
\end{minipage}\hfill
\begin{minipage}{0.32\textwidth}
    \centering
    \includegraphics[width=\textwidth]{scene_neon_corridor.jpg}
    \caption*{\small Neon corridor: moody futuristic with vertical light bars}
\end{minipage}\hfill
\begin{minipage}{0.32\textwidth}
    \centering
    \includegraphics[width=\textwidth]{scene_black_grid.jpg}
    \caption*{\small Black grid room: Tron-like with neon edge lighting}
\end{minipage}
\caption{The three input scene reference images that directed Phase 5 creative expansion. Each features chrome mannequins in surreal retail-futurism settings.}
\label{fig:scenes}
\end{figure}

\subsection{Scene Cleaning --- Mannequin Removal}

Before compositing, the Scene Cleaner agent removed existing mannequin subjects from each scene while preserving all environmental elements, especially the floating smiley face balloons.

\begin{figure}[H]
\centering
\begin{minipage}{0.32\textwidth}
    \centering
    \includegraphics[width=\textwidth]{cleaned_grid.jpg}
    \caption*{\small Cleaned white grid}
\end{minipage}\hfill
\begin{minipage}{0.32\textwidth}
    \centering
    \includegraphics[width=\textwidth]{cleaned_neon.jpg}
    \caption*{\small Cleaned neon corridor}
\end{minipage}\hfill
\begin{minipage}{0.32\textwidth}
    \centering
    \includegraphics[width=\textwidth]{cleaned_tron.jpg}
    \caption*{\small Cleaned black grid}
\end{minipage}
\caption{Cleaned scenes with mannequins removed. The Nano Banana model preserved all environmental details---floating smiley balloons, neon light bars, grid patterns, and geometric structures. The black grid room required two passes to fully remove both figures.}
\label{fig:cleaned}
\end{figure}

\subsection{Direct Scene Composites (9 Images)}

The Direct Composite agent placed our chrome mannequin wearing the Topshop dress into each of the three scene environments, generating three variations per scene.

\begin{figure}[H]
\centering
\begin{minipage}{0.32\textwidth}
    \centering
    \includegraphics[width=\textwidth]{composite_grid.jpg}
    \caption*{\small Grid room: center stance with floating smileys}
\end{minipage}\hfill
\begin{minipage}{0.32\textwidth}
    \centering
    \includegraphics[width=\textwidth]{composite_neon.jpg}
    \caption*{\small Neon corridor: emerging from darkness, rim-lit chrome}
\end{minipage}\hfill
\begin{minipage}{0.32\textwidth}
    \centering
    \includegraphics[width=\textwidth]{composite_tron.jpg}
    \caption*{\small Tron room: neon grid reflecting on chrome skin}
\end{minipage}
\caption{Direct scene composites. The garment reference (front panel) and scene image were sent together to Nano Banana, which generated a new image placing the dressed mannequin into the environment.}
\label{fig:composites}
\end{figure}

\begin{figure}[H]
\centering
\begin{minipage}{0.32\textwidth}
    \centering
    \includegraphics[width=\textwidth]{composite_grid_walk.jpg}
    \caption*{\small Grid: dynamic mid-stride, smiley foreground bokeh}
\end{minipage}\hfill
\begin{minipage}{0.32\textwidth}
    \centering
    \includegraphics[width=\textwidth]{composite_neon_profile.jpg}
    \caption*{\small Neon: side profile, dress catching teal glow}
\end{minipage}\hfill
\begin{minipage}{0.32\textwidth}
    \centering
    \includegraphics[width=\textwidth]{composite_tron_duo.jpg}
    \caption*{\small Tron: twin mannequins, both in the same dress}
\end{minipage}
\caption{Additional scene composite variations demonstrating pose and composition diversity within each environment.}
\label{fig:composites2}
\end{figure}

\subsection{Creative Riff Variations (25 Images)}

Two agents generated creative variations that pushed beyond literal scene recreation into high-concept editorial territory. Each riff maintained the dress as the sole constant while exploring radically different environments and moods.

\subsubsection{Smiley Theme Riffs}

\begin{figure}[H]
\centering
\begin{minipage}{0.24\textwidth}
    \centering
    \includegraphics[width=\textwidth]{riff_smiley_rain.jpg}
    \caption*{\tiny Smiley rain: giant balloons as weather phenomena}
\end{minipage}\hfill
\begin{minipage}{0.24\textwidth}
    \centering
    \includegraphics[width=\textwidth]{riff_underwater.jpg}
    \caption*{\tiny Underwater: smileys drifting like jellyfish}
\end{minipage}\hfill
\begin{minipage}{0.24\textwidth}
    \centering
    \includegraphics[width=\textwidth]{riff_smiley_field.jpg}
    \caption*{\tiny White void: minimalist smiley sphere field}
\end{minipage}\hfill
\begin{minipage}{0.24\textwidth}
    \centering
    \includegraphics[width=\textwidth]{riff_smiley_army.jpg}
    \caption*{\tiny Smiley army: 50+ floating faces, one mannequin}
\end{minipage}
\caption{Smiley theme riffs --- the floating smiley face motif from the white grid room scene is reimagined across diverse environments: heavy rain, underwater, infinite void, and maximalist swarm.}
\label{fig:smileys}
\end{figure}

\subsubsection{Neon and Cyberpunk Riffs}

\begin{figure}[H]
\centering
\begin{minipage}{0.24\textwidth}
    \centering
    \includegraphics[width=\textwidth]{riff_cityscape.jpg}
    \caption*{\tiny Tokyo rooftop: Blade Runner couture}
\end{minipage}\hfill
\begin{minipage}{0.24\textwidth}
    \centering
    \includegraphics[width=\textwidth]{riff_neon_laser.jpg}
    \caption*{\tiny Laser void: geometric beam framework}
\end{minipage}\hfill
\begin{minipage}{0.24\textwidth}
    \centering
    \includegraphics[width=\textwidth]{riff_hologram.jpg}
    \caption*{\tiny Holographic projection with scan lines}
\end{minipage}\hfill
\begin{minipage}{0.24\textwidth}
    \centering
    \includegraphics[width=\textwidth]{riff_grid_neon.jpg}
    \caption*{\tiny Half-white, half-black split room}
\end{minipage}
\caption{Neon/cyberpunk riffs building on the corridor and grid room themes. The cream dress provides warm organic contrast against cold digital environments.}
\label{fig:neon}
\end{figure}

\subsubsection{Nature and Elemental Riffs}

\begin{figure}[H]
\centering
\begin{minipage}{0.24\textwidth}
    \centering
    \includegraphics[width=\textwidth]{riff_greenhouse.jpg}
    \caption*{\tiny Victorian greenhouse: real chrysanthemums mirror bodice art}
\end{minipage}\hfill
\begin{minipage}{0.24\textwidth}
    \centering
    \includegraphics[width=\textwidth]{riff_cherry.jpg}
    \caption*{\tiny Cherry blossom: petals like confetti}
\end{minipage}\hfill
\begin{minipage}{0.24\textwidth}
    \centering
    \includegraphics[width=\textwidth]{riff_desert.jpg}
    \caption*{\tiny Desert: golden hour, cracked earth, amber sky}
\end{minipage}\hfill
\begin{minipage}{0.24\textwidth}
    \centering
    \includegraphics[width=\textwidth]{riff_thunderstorm.jpg}
    \caption*{\tiny Thunderstorm: clifftop, lightning, raw power}
\end{minipage}
\caption{Nature riffs. The greenhouse concept creates a direct dialogue between the botanical ink illustrations on the dress bodice and real flowers. The thunderstorm concept inverts the typical fashion editorial by embracing elemental chaos.}
\label{fig:nature}
\end{figure}

\subsubsection{Architectural and Art Riffs}

\begin{figure}[H]
\centering
\begin{minipage}{0.24\textwidth}
    \centering
    \includegraphics[width=\textwidth]{riff_cathedral.jpg}
    \caption*{\tiny Gothic cathedral: stained glass on cream fabric}
\end{minipage}\hfill
\begin{minipage}{0.24\textwidth}
    \centering
    \includegraphics[width=\textwidth]{riff_brutalist.jpg}
    \caption*{\tiny Brutalist staircase: infinite spiral ascent}
\end{minipage}\hfill
\begin{minipage}{0.24\textwidth}
    \centering
    \includegraphics[width=\textwidth]{riff_museum.jpg}
    \caption*{\tiny Museum installation: fashion as fine art}
\end{minipage}\hfill
\begin{minipage}{0.24\textwidth}
    \centering
    \includegraphics[width=\textwidth]{riff_parking.jpg}
    \caption*{\tiny Parking garage: industrial fluorescent contrast}
\end{minipage}
\caption{Architectural riffs spanning sacred (cathedral), monumental (brutalist), institutional (museum), and industrial (parking garage) spaces.}
\label{fig:architecture}
\end{figure}

\subsubsection{Surreal and Conceptual Riffs}

\begin{figure}[H]
\centering
\begin{minipage}{0.24\textwidth}
    \centering
    \includegraphics[width=\textwidth]{riff_infinity.jpg}
    \caption*{\tiny Infinity mirrors: Kusama-inspired pinstripe patterns}
\end{minipage}\hfill
\begin{minipage}{0.24\textwidth}
    \centering
    \includegraphics[width=\textwidth]{riff_giant.jpg}
    \caption*{\tiny Giant mannequin: towering over miniature city}
\end{minipage}\hfill
\begin{minipage}{0.24\textwidth}
    \centering
    \includegraphics[width=\textwidth]{riff_fragmented.jpg}
    \caption*{\tiny Fragmented mirror: shattered reflection shards}
\end{minipage}\hfill
\begin{minipage}{0.24\textwidth}
    \centering
    \includegraphics[width=\textwidth]{riff_double_exp.jpg}
    \caption*{\tiny Double exposure: dress merged with chrysanthemums}
\end{minipage}
\caption{Surreal/conceptual riffs pushing beyond conventional fashion photography into art direction territory. The double exposure concept merges the botanical bodice illustrations with real flowers.}
\label{fig:surreal}
\end{figure}

\subsubsection{Pop Culture Riffs}

\begin{figure}[H]
\centering
\begin{minipage}{0.32\textwidth}
    \centering
    \includegraphics[width=\textwidth]{riff_pop_art.jpg}
    \caption*{\small Pop art: Warhol-inspired with Campbell's-style cans}
\end{minipage}\hfill
\begin{minipage}{0.32\textwidth}
    \centering
    \includegraphics[width=\textwidth]{riff_graffiti.jpg}
    \caption*{\small Street fashion: haute couture meets graffiti walls}
\end{minipage}\hfill
\begin{minipage}{0.32\textwidth}
    \centering
    \includegraphics[width=\textwidth]{riff_vaporwave.jpg}
    \caption*{\small Vaporwave: pink/teal with Greek columns and checkered floors}
\end{minipage}
\caption{Pop culture riffs. Each places the same garment into a radically different cultural context, demonstrating the dress's versatility as a creative canvas.}
\label{fig:popculture}
\end{figure}

\subsubsection{Fashion Industry Riffs}

\begin{figure}[H]
\centering
\begin{minipage}{0.45\textwidth}
    \centering
    \includegraphics[width=\textwidth]{riff_runway.jpg}
    \caption*{\small Runway: fashion week with smiley spotlights}
\end{minipage}\hfill
\begin{minipage}{0.45\textwidth}
    \centering
    \includegraphics[width=\textwidth]{riff_ice_cave.jpg}
    \caption*{\small Ice cave: warm cream against frozen blue walls}
\end{minipage}
\caption{The runway concept brings the campaign full circle to fashion's traditional format, while the ice cave pushes into otherworldly editorial territory. Both maintain perfect garment fidelity.}
\label{fig:fashion}
\end{figure}

\subsection{Flux 2 Dev Local GPU Renders (8 Images)}

Simultaneously, the Flux 2 Local agent used the dual RTX 6000 Ada GPUs to generate editorial shots through the local ComfyUI pipeline.

\begin{figure}[H]
\centering
\begin{minipage}{0.24\textwidth}
    \centering
    \includegraphics[width=\textwidth]{flux2_warehouse.jpg}
    \caption*{\tiny Smiley warehouse}
\end{minipage}\hfill
\begin{minipage}{0.24\textwidth}
    \centering
    \includegraphics[width=\textwidth]{flux2_neon.jpg}
    \caption*{\tiny Neon alley}
\end{minipage}\hfill
\begin{minipage}{0.24\textwidth}
    \centering
    \includegraphics[width=\textwidth]{flux2_mirror.jpg}
    \caption*{\tiny Mirror room}
\end{minipage}\hfill
\begin{minipage}{0.24\textwidth}
    \centering
    \includegraphics[width=\textwidth]{flux2_subway.jpg}
    \caption*{\tiny Subway platform}
\end{minipage}
\caption{Flux 2 Dev local GPU renders. These were generated entirely on-premises using the dual RTX 6000 Ada pipeline, demonstrating that high-quality editorial imagery can be produced without cloud API dependencies. The Flux 2 model excels at photorealistic lighting and material rendering.}
\label{fig:flux2}
\end{figure}

% ============================================================
% 9. TIMING ANALYSIS AND WORLD RECORD METRICS
% ============================================================
\section{Timing Analysis and World Record Metrics}

\subsection{Timestamp-Verified Pipeline Timeline}

\begin{table}[H]
\centering
\small
\begin{tabularx}{\textwidth}{Xlrr}
\toprule
\textbf{Event} & \textbf{UTC} & \textbf{$\Delta$} & \textbf{Assets} \\
\midrule
Campaign config created & 10:20 & T+0 & --- \\
Phase 1 start (Flux 2) & 10:26 & T+6m & --- \\
Phase 1 complete & 10:32 & T+12m & 6 \\
Phase 2 complete (Nano Banana) & 10:38 & T+18m & 12 \\
Phase 3 complete (compositing) & 10:45 & T+25m & 30 \\
Phase 4 complete (Veo animation) & 10:50 & T+30m & 36 \\
Garment fidelity fix complete & 11:17 & T+57m & 60 \\
Scene riffs launched (5 agents) & 11:31 & T+71m & 60 \\
First composites landing & 11:33 & T+73m & 63 \\
22 scene riffs complete & 11:36 & T+76m & 82 \\
Remaining riffs + Flux 2 renders & 11:42+ & T+82m+ & 100+ \\
\bottomrule
\end{tabularx}
\caption{Complete pipeline timeline with verified timestamps}
\end{table}

\subsection{Asset Creation Rates}

\begin{table}[H]
\centering
\begin{tabular}{lrrr}
\toprule
\textbf{Phase} & \textbf{Assets} & \textbf{Minutes} & \textbf{Rate (img/min)} \\
\midrule
Phase 1: Base Gen (Flux 2) & 6 & 6 & 1.0 \\
Phase 2: Refinement (Nano Banana) & 6 & 6 & 1.0 \\
Phase 3: Compositing (Pillow) & 18 & 7 & 2.6 \\
Phase 4: Animation (Veo) & 6 & 5 & 1.2 \\
Phase 4b: Reskinning & 24 & 22 & 1.1 \\
Phase 5: Scene Riffs (parallel) & 45 & $\sim$12 & 3.8 \\
\midrule
\textbf{Cumulative} & \textbf{105} & \textbf{$\sim$58} & \textbf{1.8} \\
\bottomrule
\end{tabular}
\caption{Asset creation rates by phase. Peak throughput of 3.8--4.4 images/minute during parallel swarm execution.}
\end{table}

\subsection{Traditional Workflow Comparison}

\begin{tcolorbox}[metricbox, title=IRIS vs Traditional Pipeline (THG Freepik Spaces Baseline)]
\begin{center}
\begin{tabular}{lccc}
\toprule
\textbf{Metric} & \textbf{Traditional} & \textbf{IRIS} & \textbf{Multiplier} \\
\midrule
Per-image generation & 15 min & $\sim$1 min & \textbf{15$\times$} \\
Campaign (36 assets) & $\sim$4 hours & 30 min & \textbf{8$\times$} \\
With reskinning (60 assets) & $\sim$8+ hours & 57 min & \textbf{8.4$\times$} \\
With scene riffs (100+ assets) & $\sim$12+ hours & $\sim$60 min & \textbf{12.6$\times$} \\
Creative concepts per session & 6--8 & 45+ & \textbf{5.6$\times$} \\
Concurrent workflows & 1 & 5 & \textbf{5$\times$} \\
\bottomrule
\end{tabular}
\end{center}
\end{tcolorbox}

\subsection{Voice Prompt Efficiency}

\begin{table}[H]
\centering
\begin{tabular}{lr}
\toprule
\textbf{Metric} & \textbf{Value} \\
\midrule
Total voice prompts & 15 \\
Total assets generated & 100+ \\
Assets per prompt & 6.7+ \\
Average prompt length & $\sim$2 sentences \\
Total human input time & $\sim$5 minutes \\
Total autonomous execution & $\sim$60 minutes \\
Human:Machine time ratio & 1:12 \\
\bottomrule
\end{tabular}
\caption{Voice prompt efficiency metrics. 5 minutes of human creative direction produced 100+ campaign assets.}
\end{table}

% ============================================================
% 10. TECHNICAL ARCHITECTURE DETAIL
% ============================================================
\section{Technical Architecture}

\subsection{ComfyUI Multi-GPU Workflow}

\begin{figure}[H]
\centering
\begin{minipage}{0.48\textwidth}
    \centering
    \includegraphics[width=\textwidth]{flux2_rain_grid.jpg}
    \caption*{\small Flux 2 rain grid render}
\end{minipage}\hfill
\begin{minipage}{0.48\textwidth}
Two loadable ComfyUI workflow JSON files were created:

\begin{itemize}[nosep]
    \item \texttt{flux2-multigpu-campaign.json}: Base generation pipeline with \texttt{UNETLoaderMultiGPU} (cuda:0), \texttt{CLIPLoaderMultiGPU} (cuda:1), \texttt{VAELoaderMultiGPU} (cuda:1)
    \item \texttt{nano-banana-garment-reskin.json}: Garment reskinning workflow using \texttt{LoadImage} $\rightarrow$ \texttt{ImageBatch} $\rightarrow$ \texttt{GeminiImageNode}
\end{itemize}

Both workflows are loadable directly in the ComfyUI interface for re-use or modification.
\end{minipage}
\caption{Local GPU rendering via the Flux 2 Dev multi-GPU pipeline}
\end{figure}

\subsection{Key Technical Discoveries}

\begin{enumerate}[nosep]
    \item \textbf{Veo 3.1 API:} The \texttt{generateVideo} endpoint returns 404; use \texttt{predictLongRunning} instead. Video downloads require \texttt{-L} flag for HTTP redirects.
    \item \textbf{AI Text Rendering:} Nano Banana consistently misspells complex words. Programmatic rendering (Pillow) guarantees accuracy.
    \item \textbf{Multi-Panel References:} Sending a full 4-panel garment image causes grid/collage output. Single-panel crops are essential.
    \item \textbf{Garment Fidelity:} Flux 2 Dev text prompts alone cannot reliably reproduce specific garment details. Image reference via Nano Banana is required for fidelity.
    \item \textbf{Parallel Agent Throughput:} 5 simultaneous agents achieve 4.4 images/minute peak rate.
\end{enumerate}

% ============================================================
% 11. QUALITY AND BRAND COMPLIANCE
% ============================================================
\section{Quality and Brand Compliance}

The Brand Guardian agent autonomously evaluated all assets against Topshop brand criteria:

\begin{table}[H]
\centering
\begin{tabular}{lrrrr}
\toprule
\textbf{Phase} & \textbf{Assets} & \textbf{Passed} & \textbf{Failed} & \textbf{Pass Rate} \\
\midrule
Base Generation & 6 & 6 & 0 & 100\% \\
Style Refinement & 6 & 6 & 0 & 100\% \\
Compositing (v1) & 9 & 0 & 9 & 0\% \\
Compositing (final) & 18 & 18 & 0 & 100\% \\
Animation & 6 & 6 & 0 & 100\% \\
\midrule
\textbf{Total (final)} & \textbf{36} & \textbf{36} & \textbf{0} & \textbf{100\%} \\
\bottomrule
\end{tabular}
\caption{Brand Guardian QA results. The v1 compositing failure (misspelled text) was autonomously detected and fixed.}
\end{table}

% ============================================================
% 12. CONCLUSION
% ============================================================
\section{Conclusion}

The IRIS system demonstrated that autonomous AI agent swarms can dramatically accelerate fashion campaign production while maintaining creative quality and brand compliance. Key achievements:

\begin{itemize}[nosep]
    \item \textbf{100+ assets} generated from a single garment photograph
    \item \textbf{8--12.6$\times$ speedup} over the traditional 4-hour expert workflow
    \item \textbf{15 voice prompts} ($\sim$5 minutes of human input) drove the entire campaign
    \item \textbf{Autonomous error recovery:} text misspelling, garment fidelity, and multi-panel issues were identified and resolved by the system
    \item \textbf{Creative diversity:} 45+ unique editorial concepts spanning architecture, nature, pop culture, cyberpunk, surrealism, and more
    \item \textbf{Multi-engine orchestration:} seamless coordination of local GPU (Flux 2 Dev), cloud image (Nano Banana), cloud video (Veo 3.1), and programmatic rendering
\end{itemize}

This world record attempt validates the vision behind THG Ingenuity's Agentic Catwalk: AI-driven campaign generation is not merely faster than traditional workflows---it enables a qualitatively different creative process where human creative directors provide high-level voice direction while autonomous systems handle the full technical execution pipeline.

The future of fashion advertising lies at this intersection of human creativity and machine capability, and the IRIS system represents a significant step toward that future.

\vfill
\begin{center}
\textcolor{medgrey}{\small --- End of Report ---}\\[0.3cm]
\textcolor{medgrey}{\small Repository: \url{https://github.com/DreamLab-AI/THG-world-record-attempt}}\\
\textcolor{medgrey}{\small Generated: February 25, 2026}
\end{center}

\end{document}
